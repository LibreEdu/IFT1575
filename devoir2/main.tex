\documentclass{article}
\usepackage[utf8]{inputenc}
\usepackage{array}
\usepackage{amsmath}

\title{IFT1575 devoir 2}
\author{ }
\date{February 2020}

\begin{document}

\maketitle

\section{Question 1}

\begin{center}
\renewcommand{\arraystretch}{1.5}
\begin{tabular}{|>{\centering\arraybackslash}m{5mm}| >{\centering\arraybackslash}m{7mm}>{\centering\arraybackslash}m{7mm}>{\centering\arraybackslash}m{7mm}>{\centering\arraybackslash}m{7mm}>{\centering\arraybackslash}m{7mm}>{\centering\arraybackslash}m{7mm}>{\centering\arraybackslash}m{7mm}>{\centering\arraybackslash}m{5mm}|>{\centering\arraybackslash}m{5mm}|} 
 \hline
 v.d   &$x_1$&$x_2$&$x_3$&$x_4$&$x_5$&$x_6$&$x_7$&$-z$& t.d \\ 
 \hline
 $x_3$ &  3  &     &  1  & $a$ &     & -1  &     &    &  1 \\ 
 $x_2$ &  4  &  1  &     & -1  &     &  2  & $b$ &    &  6 \\ 
 $x_5$ & -1  &     &     &  2  &  1  & $c$ &  4  &    &  5 \\ 
 \hline
  $-z$ &  2  &     &     &  3  &     &  2  & $d$ & 1  &  25 \\
 \hline
\end{tabular}
\end{center}

Afin que $x_5$ soit variable de sortie, on doit avoir $min \{\frac{5}{4}, \frac{6}{b}\}$ = $\frac{5}{4}$ ce qui implique que 
$\frac{5}{4} < \frac{6}{b}$ et donc que $b < \frac{24}{5}$. On peut alors effectuer on pivot pour obtenir le tableau suivant:

\begin{center}
\renewcommand{\arraystretch}{1.5}
\begin{tabular}{|>{\centering\arraybackslash}m{5mm}| >{\centering\arraybackslash}m{7mm}>{\centering\arraybackslash}m{7mm}>{\centering\arraybackslash}m{7mm}>{\centering\arraybackslash}m{7mm}>{\centering\arraybackslash}m{7mm}>{\centering\arraybackslash}m{7mm}>{\centering\arraybackslash}m{7mm}>{\centering\arraybackslash}m{5mm}|>{\centering\arraybackslash}m{9mm}|} 
 \hline
 v.d   &$x_1$&$x_2$&$x_3$&$x_4$&$x_5$&$x_6$&$x_7$&$-z$& t.d \\ 
 \hline
 $x_3$ &  3  &     &  1  & $a$ &     & -1  &     &    &  1 \\ 
 $x_2$ &4+$\frac{b}{4}$  &  1  &     & -1-$\frac{b}{2}$  &-$\frac{b}{4}$&2-$\frac{bc}{4}$& & & 6-$\frac{5b}{4}$ \\ 
 $x_7$ & -$\frac{1}{4}$  &     &     & $\frac{1}{2}$ & $\frac{1}{4}$& $\frac{c}{4}$ &  1  &    &  $\frac{5}{4}$ \\ 
 \hline
  $-z$ &2+$\frac{d}{4}$  &     &     &  3-$\frac{d}{2}$  & -$\frac{d}{4}$& 2-$\frac{cd}{4}$&  & 1  & 25-$\frac{5d}{4}$ \\
 \hline
\end{tabular}
\end{center}

\renewcommand{\theenumi}{\alph{enumi}}
\begin{enumerate}
\item Solution optimale unique

Afin que la solution soit optimale, tous les coût doivent être positifs. Les variables $a,b,c,d$ doivent donc respecter les contraintes suivantes:\newline

(1) $2 + \frac{d}{4} > 0 \Rightarrow d > -8$ \newline
(2) $3 - \frac{d}{2} > 0 \Rightarrow d < 6$ \newline
(3) $ - \frac{d}{4} > 0 \Rightarrow d < 0$ \newline
(4) $2 - \frac{cd}{4} > 0 \Rightarrow cd < 8$ \newline

On peut donc choisir par exemple $a = 1$, $b=4$, $c=2$ et $d=-4$ pour obtenir le tableau suivant qui est optimale:

\begin{center}
\renewcommand{\arraystretch}{1.5}
\begin{tabular}{|>{\centering\arraybackslash}m{5mm}| >{\centering\arraybackslash}m{7mm}>{\centering\arraybackslash}m{7mm}>{\centering\arraybackslash}m{7mm}>{\centering\arraybackslash}m{7mm}>{\centering\arraybackslash}m{7mm}>{\centering\arraybackslash}m{7mm}>{\centering\arraybackslash}m{7mm}>{\centering\arraybackslash}m{5mm}|>{\centering\arraybackslash}m{9mm}|} 
 \hline
 v.d   &$x_1$&$x_2$&$x_3$&$x_4$&$x_5$&$x_6$&$x_7$&$-z$& t.d \\ 
 \hline
 $x_3$ &  3  &     &  1  &  1  &     & -1  &     &    &  1 \\ 
 $x_2$ &  5  &  1  &     & -3  & -1  &     &     &    &  1 \\ 
 $x_7$ & -$\frac{1}{4}$  &     &     & $\frac{1}{2}$ & $\frac{1}{4}$& $\frac{1}{2}$ &  1  &    &  $\frac{5}{4}$ \\ 
 \hline
  $-z$ &  1  &     &     &  5  &  1  &  4  &     &  1 & 30 \\
 \hline
\end{tabular}
\end{center}

\item Solution optimale qui n'est pas unique

Pour avoir une solution optimale qui n'est pas unique, on doit avoir une des variables indépendantes qui a un coût nul. En fixant par exemple $\overline{c_1}$ à zéro, on obtient la contrainte:\newline

(5) $2 + \frac{d}{4} = 0 \Rightarrow d = -8$ \newline

En plus des contraintes (2), (3), (4). On peut alors choisir $a = 1$, $b=4$, $c=2$ et $d=-8$ pour obtenir le tableau suivant qui est optimale mais non-unique car on pourrait effectuer un pivot avec $x_1$ comme variable d'entrée pour obtenir une autre solution:

\begin{center}
\renewcommand{\arraystretch}{1.5}
\begin{tabular}{|>{\centering\arraybackslash}m{5mm}| >{\centering\arraybackslash}m{7mm}>{\centering\arraybackslash}m{7mm}>{\centering\arraybackslash}m{7mm}>{\centering\arraybackslash}m{7mm}>{\centering\arraybackslash}m{7mm}>{\centering\arraybackslash}m{7mm}>{\centering\arraybackslash}m{7mm}>{\centering\arraybackslash}m{5mm}|>{\centering\arraybackslash}m{9mm}|} 
 \hline
 v.d   &$x_1$&$x_2$&$x_3$&$x_4$&$x_5$&$x_6$&$x_7$&$-z$& t.d \\ 
 \hline
 $x_3$ &  3  &     &  1  &  1  &     & -1  &     &    &  1 \\ 
 $x_2$ &  5  &  1  &     & -3  & -1  &     &     &    &  1 \\ 
 $x_7$ & -$\frac{1}{4}$  &     &     & $\frac{1}{2}$ & $\frac{1}{4}$& $\frac{1}{2}$ &  1  &    &  $\frac{5}{4}$ \\ 
 \hline
  $-z$ &     &     &     &  7  &  2  &  6  &     &  1 & 35 \\
 \hline
\end{tabular}
\end{center}

\item Un problème non borné inférieurement

Afin que le problème soit non borné inférieurement, on doit avoir une variable dont le coût est négatif et dont les coefficients dans toutes les contraintes sont négatifs. Dans ce cas, on pourra augmenter cette variable indéfiniment et donc faire diminuer l'objectif autant qu'on veut. Si on choisi par exemple la variable $x_6$, les contraintes suivantes sur $a$,$b$,$c$ et $d$ doivent être satisfaites:\newline

(6) $2 - \frac{bc}{4} < 0 \Rightarrow bc > 8$ \newline
(7) $\frac{c}{4} < 0 \Rightarrow  c < 0$ \newline
(8) $2 - \frac{cd}{4} < 0 \Rightarrow cd > 8$ \newline

 On peut alors choisir $a = 1$, $b=-8$, $c=-2$ et $d=-8$ pour obtenir le tableau suivant:
 
 \begin{center}
\renewcommand{\arraystretch}{1.5}
\begin{tabular}{|>{\centering\arraybackslash}m{5mm}| >{\centering\arraybackslash}m{7mm}>{\centering\arraybackslash}m{7mm}>{\centering\arraybackslash}m{7mm}>{\centering\arraybackslash}m{7mm}>{\centering\arraybackslash}m{7mm}>{\centering\arraybackslash}m{7mm}>{\centering\arraybackslash}m{7mm}>{\centering\arraybackslash}m{5mm}|>{\centering\arraybackslash}m{9mm}|} 
 \hline
 v.d   &$x_1$&$x_2$&$x_3$&$x_4$&$x_5$&$x_6$&$x_7$&$-z$& t.d \\ 
 \hline
 $x_3$ &  3  &     &  1  &  1  &     & -1  &     &    &  1 \\ 
 $x_2$ &  2  &  1  &     &  3  &  2  & -2  &     &    &  16 \\ 
 $x_7$ & -$\frac{1}{4}$  &     &     & $\frac{1}{2}$ & $\frac{1}{4}$& -$\frac{1}{2}$ &  1  &    &  $\frac{5}{4}$ \\ 
 \hline
  $-z$ &     &     &     &  7  &  2  & -2  &     &  1 & 35 \\
 \hline
\end{tabular}
\end{center}

\end{enumerate}

\section{Question 2}

On a le tableau initial suivant:

\begin{center}
\renewcommand{\arraystretch}{1.5}
\begin{tabular}{|>{\centering\arraybackslash}m{5mm}| >{\centering\arraybackslash}m{7mm}>{\centering\arraybackslash}m{8mm}>{\centering\arraybackslash}m{7mm}>{\centering\arraybackslash}m{7mm}>{\centering\arraybackslash}m{8mm}>{\centering\arraybackslash}m{7mm}>{\centering\arraybackslash}m{7mm}>{\centering\arraybackslash}m{8mm}>{\centering\arraybackslash}m{7mm}|>{\centering\arraybackslash}m{5mm}|} 
 \hline
 v.d &$x_{11}$&$x_{21}$ & ... &$x_{1j}$&$x_{2j}$ & ... &$x_{1n}$& $x_{2n}$&$-z$& t.d \\
 \hline
     &$a_{11}$&-$a_{11}$& ... &$a_{1j}$&-$a_{1j}$& ... &$a_{1n}$&-$a_{1n}$&    & $b_1$ \\ 
     &$a_{21}$&-$a_{21}$& ... &$a_{2j}$&-$a_{2j}$& ... &$a_{2n}$&-$a_{2n}$&    & $b_2$ \\
     &$\vdots$& $\vdots$& ... &$\vdots$& $\vdots$& ... &$\vdots$& $\vdots$&    & $\vdots$ \\ 
     &$a_{m1}$&-$a_{m1}$& ... &$a_{mj}$&-$a_{mj}$& ... &$a_{mn}$&-$a_{mn}$&    & $b_m$ \\
 \hline
  -z &$c_{1}$ &-$c_{1}$ & ... &$c_{j}$ &-$c_{j}$ & ... &$c_{n}$ &-$c_{n}$ &  1 &  \\
 \hline
\end{tabular}
\end{center}

Afin de rendre $x_{1j}$ variable indépendante, on doit avoir un 1 vis-à-vis de $x_{1j}$ dans une des contraintes. On peut supposer sans perte de généralité qu'il s'agit de la première contrainte. On doit alors diviser la première ligne du tableau par $a_{1j}$:

\begin{center}
\renewcommand{\arraystretch}{1.5}
\begin{tabular}{|>{\centering\arraybackslash}m{5mm}| >{\centering\arraybackslash}m{7mm}>{\centering\arraybackslash}m{8mm}>{\centering\arraybackslash}m{7mm}>{\centering\arraybackslash}m{7mm}>{\centering\arraybackslash}m{8mm}>{\centering\arraybackslash}m{7mm}>{\centering\arraybackslash}m{7mm}>{\centering\arraybackslash}m{8mm}>{\centering\arraybackslash}m{7mm}|>{\centering\arraybackslash}m{5mm}|} 
 \hline
 v.d &$x_{11}$&$x_{21}$ & ... &$x_{1j}$&$x_{2j}$ & ... &$x_{1n}$& $x_{2n}$&$-z$& t.d \\
 \hline
     &$\frac{a_{11}}{a_{1j}}$&-$\frac{a_{11}}{a_{1j}}$&...&1&-1& ... &$\frac{a_{1n}}{a_{1j}}$&-$\frac{a_{1n}}{a_{1j}}$&& $\frac{b_{1}}{a_{1j}}$ \\ 
     &$a_{21}$&-$a_{21}$& ... &$a_{2j}$&-$a_{2j}$& ... &$a_{2n}$&-$a_{2n}$&    & $b_2$ \\
     &$\vdots$& $\vdots$& ... &$\vdots$& $\vdots$& ... &$\vdots$& $\vdots$&    & $\vdots$ \\ 
     &$a_{m1}$&-$a_{m1}$& ... &$a_{mj}$&-$a_{mj}$& ... &$a_{mn}$&-$a_{mn}$&    & $b_m$ \\
 \hline
  -z &$c_{1}$ &-$c_{1}$ & ... &$c_{j}$ &-$c_{j}$ & ... &$c_{n}$ &-$c_{n}$ &  1 &  \\
 \hline
\end{tabular}
\end{center}

On peut maintenant effectuer un pivot pour obtenir:

\begin{center}
\begin{small}
\renewcommand{\arraystretch}{1.5}
\begin{tabular}{|>{\centering\arraybackslash}m{5mm}| >{\centering\arraybackslash}m{7mm}>{\centering\arraybackslash}m{8mm}>{\centering\arraybackslash}m{7mm}>{\centering\arraybackslash}m{7mm}>{\centering\arraybackslash}m{8mm}>{\centering\arraybackslash}m{7mm}>{\centering\arraybackslash}m{7mm}>{\centering\arraybackslash}m{8mm}>{\centering\arraybackslash}m{7mm}|>{\centering\arraybackslash}m{5mm}|}
 \hline
 v.d &$x_{11}$&$x_{21}$ & ... &$x_{1j}$&$x_{2j}$ & ... &$x_{1n}$& $x_{2n}$&$-z$& t.d \\
 \hline
$x_{1j}$&$\frac{a_{11}}{a_{1j}}$&-$\frac{a_{11}}{a_{1j}}$&...&1&-1& ... &$\frac{a_{1n}}{a_{1j}}$&-$\frac{a_{1n}}{a_{1j}}$&& $\frac{b_{1}}{a_{1j}}$ \\ 
     &$\overline{a_{21}}$&-$\overline{a_{21}}$& ... &&& ... &$\overline{a_{2n}}$&-$\overline{a_{2n}}$&     &$\overline{b_{2}}$ \\
     &$\vdots$& $\vdots$& ... &$\vdots$& $\vdots$& ... &$\vdots$& $\vdots$&    & $\vdots$ \\ 
     &$\overline{a_{m1}}$&-$\overline{a_{m1}}$& ... &&& ... &$\overline{a_{mn}}$&-$\overline{a_{mn}}$& &$\overline{b_{m}}$ \\
 \hline
  -z &$\overline{c_{1}}$&-$\overline{c_{1}}$& ... &$\overline{c_{j}}$&-$\overline{c_{j}}$& ... &$\overline{c_{n}}$&-$\overline{c_{1}}$&  1 & $\overline{z}$ \\
 \hline
\end{tabular}
\end{small}
\end{center}

Dans ce tableau, tous les coefficients de $x_{2j}$ sont zéros sauf dans la première ligne et donc il est impossible de faire entrer $x_{2j}$ dans la base sans faire sortir $x_{ij}$

Ceci démontre qu'il est impossible d'avoir $x_{ij}$ et $x_{2j}$ dans la base en même temps.

\section{Question 3}

\renewcommand{\theenumi}{\alph{enumi}}
\begin{enumerate}

\item Produire directement le tableau optimal à partir des informations fournies

La matrice des coefficients est la suivante:

\begin{center}
\begin{math}
A = 
\begin{pmatrix}
1 & 2 & 3 & 0\\
2 & 1 & 5 & 0\\
1 & 2 & 1 & 1
\end{pmatrix}
\end{math}
\end{center}

On peut alors obtenir les coefficient dans le tableau final en calculant $B^{-1}A$:

\begin{center}
\begin{math}
B^{-1}A =
\begin{pmatrix}
\frac{2}{3} & -\frac{1}{3} & 0\\[5pt]
-\frac{1}{3} & \frac{2}{3} & 0\\[5pt]
-1 & 0 & 1
\end{pmatrix}
\begin{pmatrix}
1 & 2 & 3 & 0\\[5pt]
2 & 1 & 5 & 0\\[5pt]
1 & 2 & 1 & 1
\end{pmatrix}
=
\begin{pmatrix}
0 & 1 & -\frac{1}{3} & 0\\[5pt]
1 & 0 &  \frac{7}{3} & 0\\[5pt]
0 & 0 &       -2     & 1
\end{pmatrix}
\end{math}
\end{center}

On peut aussi trouver les termes de droite en calculant $B^{-1}b$:

\begin{center}
\begin{math}
B^{-1}b =
\begin{pmatrix}
\frac{2}{3} & -\frac{1}{3} & 0\\[5pt]
-\frac{1}{3} & \frac{2}{3} & 0\\[5pt]
-1 & 0 & 1
\end{pmatrix}
\begin{pmatrix}
15\\[5pt]
20\\[5pt]
20
\end{pmatrix}
=
\begin{pmatrix}
\frac{10}{3}\\[5pt]
\frac{25}{3}\\[5pt]
5
\end{pmatrix}
\end{math}
\end{center}

Il reste à calculer $\overline{c_3}$ et $z$. Pour ce faire, on a besoin de $C_B$, on regardant $B^{-1}A$, on voit que:

\begin{center}
\begin{math} 
X_B =
\begin{pmatrix}
x_2\\
x_1\\
x_4
\end{pmatrix}
\end{math}
\end{center}

Et donc $C_B$ est composé des coûts des variables $x_2$,$x_1$ et $x_4$:
\begin{center}
\begin{math}
C_B = 
\begin{pmatrix}
-2\\
-1\\
1
\end{pmatrix}
\end{math}
\end{center}

Avec, $C_B$ on peut calculer $\pi^{T}$:

\begin{center}
\begin{math}
\pi^{T} = C_B^TB^{-1} =
\begin{pmatrix}
-2 & -1 & 1
\end{pmatrix}
\begin{pmatrix}
\frac{2}{3} & -\frac{1}{3} & 0\\[5pt]
-\frac{1}{3} & \frac{2}{3} & 0\\[5pt]
-1 & 0 & 1
\end{pmatrix}
=
\begin{pmatrix}
-2 & 0 & 1
\end{pmatrix}
\end{math}
\end{center}

Et donc:

\begin{center}
\begin{math}
\overline{c_3} = c_3 - \pi^Ta_{.3} = -3 - 
\begin{pmatrix}
-2 & 0 & 1
\end{pmatrix}
\begin{pmatrix}
3\\
5\\
1
\end{pmatrix}
= -3 + 5 = 2
\end{math}
\end{center}

Finalement, on peut calculer z:

\begin{center}
\begin{math}
z = -\pi^Tb = - 
\begin{pmatrix}
-2 & 0 & 1
\end{pmatrix}
\begin{pmatrix}
15\\
20\\
20
\end{pmatrix}
= 10
\end{math}
\end{center}
On obtient alors le tableau optimal suivant:

\begin{center}
\renewcommand{\arraystretch}{1.5}
\begin{tabular}{|>{\centering\arraybackslash}m{5mm}| >{\centering\arraybackslash}m{7mm}>{\centering\arraybackslash}m{7mm}>{\centering\arraybackslash}m{7mm}>{\centering\arraybackslash}m{7mm}>{\centering\arraybackslash}m{7mm}|>{\centering\arraybackslash}m{7mm}|} 
 \hline
 v.d   &$x_1$&$x_2$& $x_3$  &$x_4$&$-z$& t.d \\ 
 \hline
 $x_2$ &     &  1  &-$\frac{1}{3}$&     &    & $\frac{10}{3}$   \\ 
 $x_1$ &  1  &     &$\frac{7}{3}$ &     &    & $\frac{25}{3}$ \\ 
 $x_4$ &     &     &     -2     &  1  &    & 5 \\ 
 \hline
  $-z$ &     &     &      2     &     &  1 &  10 \\
 \hline
\end{tabular}
\end{center}

\item La base optimale demeure-t-elle optimale si le coût de la variable $x_3$ dans la formulation initiale du problème est égal à -5?

Changer le coût de $x_3$ affecte seulement $\overline{c_3}$. Le nouvelle valeure est:

\begin{center}
\begin{math}
\overline{c_3} = c_3 - \pi^Ta_{.3} = -5 - 
\begin{pmatrix}
-2 & 0 & 1
\end{pmatrix}
\begin{pmatrix}
3\\
5\\
1
\end{pmatrix}
= -5 + 5 = 0
\end{math}
\end{center}

Donc comme $\overline{c_3} \geq 0$, la base demeure optimale. Par contre, la solution n'est plus unique car on pourrait effectuer un pivot en utilisant $x_3$ comme variable d'entrée sans changer la valeure de l'objectif.

\item La base optimale demeure-t-elle optimale si les termes de droite dans la formulation initiale du problème sont égaux à 5, 10 et 10?

Modifier b n'affecte pas les coefficients ni les coût. Seulement les termes de droites doivent être recalculé:

\begin{center}
\begin{math}
B^{-1}b =
\begin{pmatrix}
\frac{2}{3} & -\frac{1}{3} & 0\\[5pt]
-\frac{1}{3} & \frac{2}{3} & 0\\[5pt]
-1 & 0 & 1
\end{pmatrix}
\begin{pmatrix}
5\\[5pt]
10\\[5pt]
10
\end{pmatrix}
=
\begin{pmatrix}
0\\[5pt]
5\\[5pt]
5
\end{pmatrix}
\end{math}
\end{center}

Comme $B^{-1}b \geq 0$, la solution est réalisable et donc la base demeure optimale.

\item Appliquer une série de transformations linéaires au tableau correspondant à la formulation initiale du problème afin d'obtenir une forme appropriée pour l’algorithme du simplexe

\begin{center}
\renewcommand{\arraystretch}{1.5}
\begin{tabular}{|>{\centering\arraybackslash}m{5mm}| >{\centering\arraybackslash}m{7mm}>{\centering\arraybackslash}m{7mm}>{\centering\arraybackslash}m{7mm}>{\centering\arraybackslash}m{7mm}>{\centering\arraybackslash}m{7mm}|>{\centering\arraybackslash}m{7mm}|} 
 \hline
 v.d   &$x_1$&$x_2$&$x_3$&$x_4$&$-z$& t.d  \\ 
 \hline
 $x_2$ &  1  &  2  &  3  &     &    &  15  \\ 
 $x_1$ &  2  &  1  &  5  &     &    &  20 \\ 
 $x_4$ &  1  &  2  &  1  &  1  &    &  20 \\ 
 \hline
  $-z$ & -1  & -2  & -3  &  1  &  1 &    \\
 \hline
\end{tabular}

 $L_1 \to \frac{L_1}{2}$ \newline

\renewcommand{\arraystretch}{1.5}
\begin{tabular}{|>{\centering\arraybackslash}m{5mm}| >{\centering\arraybackslash}m{7mm}>{\centering\arraybackslash}m{7mm}>{\centering\arraybackslash}m{7mm}>{\centering\arraybackslash}m{7mm}>{\centering\arraybackslash}m{7mm}|>{\centering\arraybackslash}m{7mm}|} 
 \hline
 v.d   &$x_1$&$x_2$&$x_3$&$x_4$&$-z$& t.d  \\ 
 \hline
 $x_2$ &  $\frac{1}{2}$  &  1  &  $\frac{3}{2}$  &     &    &  $\frac{15}{2}$  \\ 
 $x_1$ &  2  &  1  &  5  &     &    &  20 \\ 
 $x_4$ &  1  &  2  &  1  &  1  &    &  20 \\ 
 \hline
  $-z$ & -1  & -2  &  -3  &  1  &  1 &    \\
 \hline
\end{tabular}


$L_2 \to L_2 - L_1$ \newline
$L_3 \to L_3 - 2L_1$ \newline
$L_4 \to L_4 + 2L_1$ \newline

\renewcommand{\arraystretch}{1.5}
\begin{tabular}{|>{\centering\arraybackslash}m{5mm}| >{\centering\arraybackslash}m{7mm}>{\centering\arraybackslash}m{7mm}>{\centering\arraybackslash}m{7mm}>{\centering\arraybackslash}m{7mm}>{\centering\arraybackslash}m{7mm}|>{\centering\arraybackslash}m{7mm}|} 
 \hline
 v.d   &$x_1$&$x_2$&$x_3$&$x_4$&$-z$& t.d  \\ 
 \hline
 $x_2$ &  $\frac{1}{2}$  &  1  &  $\frac{3}{2}$  &     &    &  $\frac{15}{2}$  \\ 
 $x_1$ &  $\frac{3}{2}$  &     &  $\frac{7}{2}$  &     &    &  $\frac{25}{2}$ \\ 
 $x_4$ &     &     &  -2  &  1  &    &  5 \\ 
 \hline
  $-z$ &     &     &     &  1  &  1  & 15     \\
 \hline
\end{tabular}

$L_2 \to \frac{2}{3}L_2$

\renewcommand{\arraystretch}{1.5}
\begin{tabular}{|>{\centering\arraybackslash}m{5mm}| >{\centering\arraybackslash}m{7mm}>{\centering\arraybackslash}m{7mm}>{\centering\arraybackslash}m{7mm}>{\centering\arraybackslash}m{7mm}>{\centering\arraybackslash}m{7mm}|>{\centering\arraybackslash}m{7mm}|} 
 \hline
 v.d   &$x_1$&$x_2$&$x_3$&$x_4$&$-z$& t.d  \\ 
 \hline
 $x_2$ &  $\frac{1}{2}$  &  1  &  $\frac{3}{2}$  &     &    &  $\frac{15}{2}$  \\ 
 $x_1$ &  1  &     &  $\frac{7}{3}$  &     &    &  $\frac{25}{3}$ \\ 
 $x_4$ &     &     &  -2  &  1  &    &  5 \\ 
 \hline
  $-z$ &     &     &      &  1  &  1  & 15     \\
 \hline
\end{tabular}

$L_1 \to L_1 - \frac{1}{2}L_2$ \newline
$L_4 \to L_4 - L_3$

\renewcommand{\arraystretch}{1.5}
\begin{tabular}{|>{\centering\arraybackslash}m{5mm}| >{\centering\arraybackslash}m{7mm}>{\centering\arraybackslash}m{7mm}>{\centering\arraybackslash}m{7mm}>{\centering\arraybackslash}m{7mm}>{\centering\arraybackslash}m{7mm}|>{\centering\arraybackslash}m{7mm}|} 
 \hline
 v.d   &$x_1$&$x_2$&$x_3$&$x_4$&$-z$& t.d  \\ 
 \hline
 $x_2$ &     &  1  &  $\frac{1}{3}$  &     &    &  $\frac{10}{3}$  \\ 
 $x_1$ &  1  &     &  $\frac{7}{3}$  &     &    &  $\frac{25}{3}$ \\ 
 $x_4$ &     &     &  -2  &  1  &    &  5 \\ 
 \hline
  $-z$ &     &     &  2   &     &  1  & 10     \\
 \hline
\end{tabular}

\end{center}

\end{enumerate}
\end{document}
